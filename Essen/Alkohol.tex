\section{酒桌文化}
\par
中国酒桌文化是中国特有的文化,历史几乎与人的历史一样久远,早在汉字成熟之前,中国人就已经掌握了酿酒技术。很多典籍中都有关于酒和饮酒文化的记载,《诗经》中有20多处提到酒,酒被赋予了礼仪、社交、休闲等含义,体现了特定的宗法秩序以及人伦关系。还有很多典籍专门讲酒,如西周的《酒诰》,西汉的《酒赋》《酒箴》,东晋的《酒诫》和初唐的《酒经》《酒谱》等等。可见,酒很早就成了中国文化的重要元素,酒文化深入中国人的血脉深处,影响深远。
\par
中国劝酒方式有文敬,回敬,互敬,代饮,罚酒.在应酬方面,交易双方通常会互相劝酒,这是对双方的一种试探;而在领导劝酒上,也是这种心理的体现,劝酒者通过观察你是否服从他要你继续饮酒的指令,观察你能不能为了“场面”而强迫自己喝酒,从而来判断你对其的服从程度。中国政界和商界的劝酒文化,要实现两个目的,它们是服从性测试和诚意测试。维系一段关系是需要付出代价的,醉酒就是这种代价。醉酒后的丑态是一种小剂量的抵押物,在人和人之间还不能完全信任,但又需要建立合作关系的时候。而德国人饮酒十分有讲究,酒一般分为三类即开胃酒,佐餐酒,饭后酒。德国人吃饭时很注重饭菜与酒的搭配。如果吃的是鱼排,就一定会喝白葡萄酒;而如果吃的是牛排,就一定会喝红葡萄酒。德国人的餐桌上较为安静,觥筹交错,却不见喧闹。为表达热情,中国人常会劝酒,但是这种现象在德国人的酒桌上是看不到的。中国人劝酒多带有目的性,而德国人酒桌上一般交友聊家常。