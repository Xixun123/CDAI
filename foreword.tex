 \vspace*{3cm}
\begin{minipage}{\linewidth}
    \Huge\heiti\center{封面图片} \par
\end{minipage}

 \vspace*{3cm}

\par
勃兰登堡门位于德国首都柏林的市中心,最初是柏林城墙的一道城门,因通往勃莱登堡而得名。现在保存的勃莱登堡门是一座古典复兴建筑,由普鲁士国王腓特烈·威廉二世下令于1788年至1791年间建造,以纪念普鲁士在七年战争取得的胜利。
\cleardoublepage

\begin{minipage}{\linewidth}
\Huge\heiti\center{关于本书} \par
\end{minipage}

 \vspace*{3cm}
 %\normalfont\normalsize
 %\raggedright
 %\indent
 中国和德国都是世界上具有重大影响力的国家,随着传播 通讯技术的改进,交通技术的进步和经济的高度全球化,两国的合作越来越频繁。 然而,由于文化背景的不同,两国人民在交往的过程中不能够相互理解,,导致交际不能顺利进行。针对中德工程师学院和德国的应用技术大学之间日益频繁的交流活动,作为中德工程师学院的学生我们希望将一些所闻所知的一些信息整理、归类,尽可能为两国的学生做一些提示。本书不求包罗万象,但求准确,详尽,有趣。

\par
 在上学期,我们小组成员在项目发起人Frau Schneider的引领和指导下,为了增进我院及合作院校中、德两国学生间的文化认同感,减轻乃至消除文化差异所带来的沟通障碍,同时也为了我院新生能够更快适应和融入国际化的教学方式,多元化的人文环境,开展了一系列关于中德文化差异的研讨会、采访、问卷调查和现场调研。我们最终决定将项目的视角聚焦在七个文化主题上:服饰、饮食、体育、工作、交通、文化、卫生,以Wiki为载体,以百科全书式的写作手法,从学生的视角出发,独特地展现出我们对于异国文化的认识与思考。在项目期间,我们充分运用所学项目管理的知识,通过Projekt Auftrag, Zeitplan, PSP等项目管理工具控制项目进程;通过查阅相关资料、与德国师生互相交流讨论,并根据实际的生活体验,确定了各项主题的具体内容,并定期向同学与老师展示工作成果。项目的最后,我们的成果是喜人的,我们的中德文化差异百科全书在学院官网上线,向中德师生展现了中德文化大花园的一隅,虽然项目仍有不少改进的空间,但是我们希望通过这个学期的共同努力,把中德文化差异百科这座文化桥梁打造的更加坚固、优美。
 相比于上学期, 这学期我们将从中德两国人民的日常行为差异入手,深入分析每个差异代表的文化内涵。我们希望通过这些方法,使中德两国国学生增进理解,弥合文化差异的鸿沟。
 \par
 在中国有一个小故事,讲的是汉朝的时候,在西南方有个名叫夜郎的小国家,它虽然是一个独立的国家,可是国土很小,百姓也少,物产更是少得可怜。但是由于邻近地区以夜郎这个国家最大,从没离开过国家的夜郎国国王就以为自己统治的国家是全天下最大的国家。有一天,夜郎国国王与部下巡视国境的时候,他指着前方问说:“这里哪个国家最大呀?”部下们为了迎合国王的心意,于是就说:“当然是夜郎国最大啰!”走着走着,国王又抬起头来、望着前方的高山问说:“天底下还有比这座山更高的山吗?”部下们回答说:“天底下没有比这座山更高的山了。”后来,他们来到河边,国王又问:“我认为这可是世界上最长的河川了。”部下们仍然异口同声回答说:“大王说得一点都没错。”从此以后,无知的国王就更相信夜郎是天底下最大的国家。有一次,汉朝派使者来到夜郎,途中先经过夜郎的邻国滇国,滇王问使者:“汉朝和我的国家比起来哪个大?”使者一听吓了一跳,他没想到这个小国家,竟然无知的自以为能与汉朝相比。却没想到后来使者到了夜郎国,骄傲又无知的国王因为不知道自己统治的国家只和汉朝的一个县差不多大,竟然不知天高地厚也问使者:“汉朝和我的国家哪个大?”。
 在如今全球化的背景下,中国青年更应该睁眼看世界,走出国门拥抱世界。德国作为世界上数一数二的大国。德国文化更是不可忽视的。面对中德两国日益平凡的交流,德国青年也是了解中国文化值得的。总之,这本小册子希望可以作为两国学生了解对方文化的起点。作为中国学生我们对德国文化没有更深入的了解,在此也欢迎正在看此书的你,提出宝贵的建议。
\vspace{\baselineskip}
\begin{flushright}\noindent
杭州, \today \hfill 项目组4\\
\end{flushright}
